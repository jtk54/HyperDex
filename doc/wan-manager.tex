\section{Georeplication with HyperDex}
\label{sec:georepl}

HyperDex has a new feature called "georeplication" available to increase the
fault-tolerance of a typical HyperDex deployment to protect against catastrophic
datacenter events, such as natural disasters or power outages. Suppose a user
has a HyperDex deployment running in Virginia that serves all of the
\code{get}, \code{put}, etc operations for the user's service. This cluster
is scalable and fault tolerant in the case of a few machines failing and
rebooting and/or being replaced; however, if the whole datacenter were
disabled, the cluster would obviously stop running.

Using georeplication in HyperDex, the user could have both the
"primary" Virginia cluster running and a secondary "backup" cluster
running in Arkansas that mirrors all the \code{put} operations
that are performed at the primary in Virginia. There is a discrete
time difference between the \code{put}s reflected in the primary
and the \code{put}s reflected in the backup, but in the limit all
\code{put}s to the primary will be reflected in the backup.

\section{Starting the Georeplicator}
\label{sec:startgeo}

To set up a primary cluster to be georeplicated, we first have to
have the primary cluster running. We start the coordinator
and a single (or more) daemon with:

\begin{consolecode}
hyperdex coordinator -f -l 127.0.0.1 -p 1982
\end{consolecode}

and

\begin{consolecode}
hyperdex daemon -f --listen=127.0.0.1 --listen-port=2012 \
                   --coordinator=127.0.0.1 --coordinator-port=1982 --data=/path/to/data
\end{consolecode}

This will be our primary cluster. We then have to initialize our backup cluster
as well by doing:

\begin{consolecode}
hyperdex coordinator -f -l 127.0.0.1 -p 1984
\end{consolecode}

and

\begin{consolecode}
hyperdex daemon -f --listen=127.0.0.1 --listen-port=2014 \
                   --coordinator=127.0.0.1 --coordinator-port=1984 --data=/path/to/data
\end{consolecode}

This is our backup cluster. For real georeplication, we would want this cluster
to be geographically separate from the primary, but for the purposes of a demo
they can both be run on localhost.

All HyperDex clusters started in this
manner are actually in primary mode, meaning that the code that pulls
\code{put}s from a primary and replicates them is not running yet. Hence
both of these clusters are primary clusters so far. Now, let's point our backup
cluster to the primary we want to georeplicate and have it pull the \code{put}s!
We do this by running the "hyperdex-set-backup-cluster" tool:

\begin{consolecode}
  hyperdex-set-backup-cluster -h 127.0.0.1 -p 1984 -r 127.0.0.1 -t 1982
\end{consolecode}

This tool contacts the backup cluster coordinator (port 1984) and tells
it to start pulling updates from the primary cluster (port 1982) and
replicate them. At this point we have the backup cluster set up to
replicate all the changes to the primary!

\section{Datacenter Catastrophe}
\label{sec:catastrophe}

As time moves along, the backup cluster we set up in the previous
section will get closer and closer to an exact replica of the primary.
Suppose now, that a major catastrophic event occurs to the primary,
taking the entire cluster out. At this point, we can move the backup cluster
to be the primary cluster (and no longer pull updates from anyone else) by
running:

\begin{consolecode}
  hyperdex-set-primary-cluster -h 127.0.0.1 -p 1984
\end{consolecode}

which sets the cluster at port 1984 to be a primary cluster. The user
running HyperDex would also have to redirect all his HyperDex clients to point
to port 1984 to make their \code{get}s, \code{put}s, etc for the conversion
to be complete.

\section{Changing Backup Affinity}
\label{sec:affinity}

Suppose now we had another primary cluster running at port 1986 with:

\begin{consolecode}
hyperdex coordinator -f -l 127.0.0.1 -p 1986
\end{consolecode}

and

\begin{consolecode}
hyperdex daemon -f --listen=127.0.0.1 --listen-port=2016 \
                   --coordinator=127.0.0.1 --coordinator-port=1986 --data=/path/to/data
\end{consolecode}

along with our other primary and backup clusters. Also suppose we already
had the backup cluster (port 1984) pulling changes from port 1982, but instead
we wanted to start backing up the cluster at port 1986 instead for some reason
(higher priority data, etc). We could use the "hyperdex-set-backup-affinity"
tool to redirect the backup to forget about the primary it was backing up
and to back up another primary instead. To do this, we run:

\begin{consolecode}
  hyperdex-set-backup-affinity -h 127.0.0.1 -p 1984 -r 127.0.0.1 -t 1986
\end{consolecode}

This tool contacts the backup cluster coordinator at port 1984 and tells it
to starting backing up the cluster at port 1986 instead of the cluster
it is currently backing up.

\section{Georeplication Conclusion}
\label{sec:geoconc}

The tools given allow users to flexibly and reliably georeplicate HyperDex clusters
to their needs, allowing performant, reliable, scalable, and linearizable data
storage at their fingertips.
